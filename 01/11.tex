\subsection{%
  Определения непрерывной функции и ее локальные свойства.%
} \label{sec:01-11}

\begin{important}
  Запись \(f(x) \iscont{\segment{a}{b}}\) означает, что функция \(f(x)\)
  непрерывна на отрезке \(\segment{a}{b}\).
  
  Запись \(f(x) \iscontd{x_0}\) означает, что функция \(f(x)\)
  непрерывна в точке \(x_0\).
\end{important}

\begin{definition} \label{def:cont-func-1}
  \begin{equation*}
    f(x) \iscontd{x_0}
    \bydef
    \lim_{x \to x_0} f(x) = f(x_0)
  \end{equation*}
\end{definition}

\begin{definition} \label{def:cont-func-2}
  \begin{equation*}
    f(x) \iscontd{x_0}
    \bydef
    \Delta x_0 \to 0 \implies \Delta y(x_0) \to 0
  \end{equation*}
\end{definition}

\begin{theorem} 
  Определения \ref{def:cont-func-1} и \ref{def:cont-func-2} равносильны.
\end{theorem}

\begin{proof}
  Рассмотрим предел \(\display{\lim_{x \to x_0} f(x) = f(x_0)}\).Если \(x \to
  x_0\), то \(\Delta x \to 0\). Получаем равносильный предел.

  \begin{equation*}
    \begin{aligned}
      \lim_{\Delta x \to 0} f(x) = f(x_0)
    \\
      \lim_{\Delta x \to 0} \under{\prh{f(x) - f(x_0)}}{\Delta y} = 0
    \\
      \lim_{\Delta x \to 0} \Delta y = 0
    \end{aligned}
  \end{equation*}

  Это равносильно тому, что \(\Delta y \to 0\) при \(\Delta x \to 0\).
\end{proof}

\begin{definition}
  Функция называется непрерывной на множестве, если она непрерывна в каждой
  точке этого множества.
\end{definition}

\begin{definition}
  Функция называется непрерывной справа/слева в точке \(x_0\), если
  соответствующий односторонний предел равен значению функции в точке \(x_0\).
\end{definition}

\begin{theorem}
  Пусть \(f(x)\) и \(g(x)\) непрерывны в точке \(a\), тогда следующие функции
  также непрерывны в точке \(a\)

  \begin{enumerate}
  \item
    \(c f(x) \given \forall c \in \RR\)

  \item
    \(f(x) + g(x)\)
  
  \item
    \(f(x) \cdot g(x)\)
  
  \item
    \(\frac{f(x)}{g(x)} \qquad (g(x) \neq 0)\)
  \end{enumerate}
\end{theorem}

\begin{proof}
  Эти свойства доказываются с помощью аналогичных свойств пределов. Например,
  докажем, что \(f(x) \cdot g(x)\) также непрерывна. Рассмотрим предел
  \(\display{\lim_{x \to a}(f(x) \cdot  g(x))}\), по свойству пределов имеем

  \begin{equation*}
    \begin{aligned}
      \lim_{x \to a}(f(x) \cdot  g(x))
      = \lim_{x \to a} f(x) \cdot \lim_{x \to a} g(x)
    \\
      \begin{rcases}
        f(x) \iscontd{a} \implies \lim_{x \to a} f(x) = f(a) \\
        g(x) \iscontd{a} \implies \lim_{x \to a} g(x) = g(a)
      \end{rcases}
      \implies
      \lim_{x \to a}(f(x) \cdot  g(x)) = f(a) \cdot  g(a)
    \end{aligned}
  \end{equation*}

  Таким образом функция \(f(x) \cdot g(x)\) также непрерывна в точке \(a\) по
  определению.
\end{proof}

\begin{theorem}
  \begin{equation*}
    \begin{rcases}
      g(x) \iscontd{x_0} \\
      f(g) \iscontd{g_0 = g(x_0)}
    \end{rcases}
    \implies
    f(g(x)) \iscontd{x_0}
  \end{equation*}
\end{theorem}

\begin{proof}
  Т.к. функции \(f(g)\) и \(g(x)\) непрерывны в точках \(g_0\) и \(x_0\)
  соответственно, то их пределы в этих точках равны значениям функций в этих
  точках, распишем пределы по определению.

  \begin{equation*}
    \begin{aligned}
      f(g) \colon \qquad
        \forall \epsilon_1 > 0 \exists \delta_1 > 0 \given
        \forall g \colon 0 < \abs{g - g_0} < \delta_1 \implies
        \abs{f(g) - f(g_0)} < \epsilon_1
      \\
      g(x) \colon \qquad
        \forall \epsilon_2 > 0 \exists \delta_2 > 0 \given
        \forall x \colon 0 < \abs{x - x_0} < \delta_2 \implies
        \abs{g(x) - g(x_0)} < \epsilon_2
    \end{aligned}
  \end{equation*}

  Возьмем произвольное \(\epsilon_1\), а \(\epsilon_2\) пусть будет равно
  \(\delta_1\). Тогда из второй строчки получаем, что

  \begin{equation*}
    \forall \epsilon_1 > 0 \exists \delta_1 > 0, \delta_2 > 0 \given
      \forall x \colon 0 < \abs{x - x_0} < \delta_2,
      \abs{g(x) - g(x_0)} < \delta_1
  \end{equation*}
  
  Т.к. \(\abs{g(x) - g(x_0)} < \delta_1\) и учитывая то, что \(g(x) - g(x_0) = g
  - g_0\), подставим получившееся выражение в определение предела для \(f(x)\).
  
  \begin{equation*}
    \forall \epsilon_1 > 0 \exists \delta_1 > 0, \delta_2 > 0 \given
    \forall x 0 < \abs{x - x_0} < \delta_2 \implies
    \abs{f(g(x)) - f(g(x_0))} < \epsilon_1
  \end{equation*}
  
  Это определение предела функции \(f(g(x))\) в точке \(x_0\). Этот предел равен
  \(f(g(x_0))\), значит функция \(f(g(x))\) непрерывна в точке \(x_0\).
\end{proof}

\begin{theorem}
  \begin{equation*}
    \begin{rcases}
      f(x) \iscontd{x_0} \\
      \exists f^{-1}(x)
    \end{rcases}
    \implies
    f^{-1}(x) \iscontd{x_0}
  \end{equation*}
\end{theorem}
