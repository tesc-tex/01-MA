\subsection{%
  Теоремы о дифференцируемых функциях. Теорема Лагранжа.%
}

\begin{theorem}[Лагранжа]
  \begin{equation*}
    f(x) \isdiff{\segment{a}{b}}
    \implies
    \exists \xi \in \interval{a}{b} \given
    \frac{f(b) - f(a)}{b - a} = f'(\xi)
  \end{equation*}
\end{theorem}

\begin{proof}
  Рассмотрим вспомогательную функцию

  \begin{equation*}
    \phi(x) = f(x) (b - a) - x \prh[\Big]{f(b) - f(a)}
  \end{equation*}

  которая удовлетворяет условиям теоремы Ролля.

  \begin{enumerate}
    \item
      \(\phi(x) \isdiff{\segment{a}{b}}\) как линейная комбинация функций
      дифференцируемых на \(\segment{a}{b}\).
    
    \item
      Она принимает равные значения на концах отрезка \(\segment{a}{b}\).
  
      \begin{enumerate}
      \item
        \(\phi(a) = f(a) b - f(a) a - f(b) a + f(a) a = f(a) b - f(b) a\)
      
      \item
        \(\phi(b) = f(b) b - f(b) a - f(b) b + f(a) b = f(a) b - f(b) a\)
      \end{enumerate}
    \end{enumerate}
  
  Таким образом по теореме Ролля

  \begin{equation*}
    \begin{aligned}
      \exists \xi \in \interval{a}{b} \given \phi'(\xi) = 0
    \\
      \phi'(\xi) = f'(\xi)(b - a) - (f(b) - f(a)) = 0
    \\
      f'(\xi) = \frac{f(b) - f(a)}{b - a}
    \end{aligned}
  \end{equation*}
\end{proof}

\begin{remark}
  Полученная формула \(\display{\frac{f(b) - f(a)}{b - a} = f'(\xi)}\) также
  называется формулой конечных приращений.
\end{remark}

Геометрический смысл теоремы Лагранжа заключается в том, что На интервале
\(\interval{a}{b}\) найдется точка \(\xi\), в которой касательная параллельна
хорде \(AB\) (\figref{01_25_01}).

Механический смысл теоремы Лагранжа заключается в том, что если тело двигалось с
переменной скоростью и при этом \(\display{\vec{v}_{\text{сред}} = \tg \phi =
\frac{B - A}{\Delta t}}\), то найдется точка \(t_0\), в которой
\(\display{\vec{v}_{\text{сред}} = \vec{v}_{\text{мгн}}}\) (\figref{01_25_02}).

\gallerydouble
  {01_25_01}{Геометрический смысл теоремы Лагранжа}
  {01_25_02}{Механический смысл теоремы Лагранжа}

