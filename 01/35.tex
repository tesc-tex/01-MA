\subsection{%
  Вторые производные функции двух переменных. Равенство смешанных производных.%
}

\begin{definition}
  Если первые производные функции двух переменных дифференцируемы, то определены
  вторые производные.

  \begin{equation*}
    \begin{aligned}
      z''_{xx} = \frac{\partial^2 z}{\partial x^2}
      \qquad
      z''_{xy} = \frac{\partial^2 z}{\partial x \partial y}
    \\
      z''_{yx} = \frac{\partial^2 z}{\partial y \partial x}
      \qquad
      z''_{yy} = \frac{\partial^2 z}{\partial y^2 }
    \end{aligned}  
  \end{equation*}

  Производные \(z''_{xx}\) и \(z''_{yy}\) называются чистыми производными.
  Производные \(z''_{xy}\) и \(z''_{yx}\) называются смешанными производными.
\end{definition}

\begin{theorem}[Шварца]
  Если смешанные производные непрерывны, то они равны.
\end{theorem}

\begin{proof}
  Рассмотрим две вспомогательные функции.

  \begin{equation*}
    \begin{aligned}
      W = \frac{f(x_0 + \Delta x, y_0 + \Delta y) - f(x_0 + \Delta x, y_0)
        - f(x_0, y_0 + \Delta y) + f(x_0 , y_0)}{\Delta x \Delta y}
    \\
      \phi(x) = \frac{f(x, y_0 + \Delta y) - f(x, y_0)}{\Delta y}
    \end{aligned}
  \end{equation*}

  Выразим \(W\) с помощью функции \(\phi(x)\), получим \(\display{W =
  \frac{\phi(x_0 + \Delta x) - \phi(x_0)}{\Delta x}}\). Т.к. функция \(\phi(x)\)
  дифференцируема (как разность дифференцируемых функций), значит можно
  воспользоваться теоремой Лагранжа, т.е.

  \begin{equation*}
    W = \phi'(\xi_1) = \frac{f'_x(\xi_1, y_0 + \Delta y)
      - f'_x(\xi_1, y_0)}{\Delta y}
    \qquad
    (\xi_1 \in \interval{x_0}{x_0 + \Delta x})
  \end{equation*}

  Введём вспомогательную функцию \(h(y) = f'_x(\xi_1, y)\) и еще раз применим
  теорему Лагранжа.

  \begin{equation*}
    \begin{aligned}
      W = \frac{h(y_0 + \Delta y) - h(y_0)}{\Delta y}
    \\
      W = h'(\zeta_1) = f''_{xy}(\xi_1, \zeta_1)
      & \qquad
      (\zeta_1 \in \interval{y_0}{y_0 + \Delta y})
    \end{aligned}
  \end{equation*}

  Аналогично можно рассмотреть вспомогательную функцию \(\display{\psi(y) =
  \frac{f(x_0 + \Delta x, y) - f(x_0, y)}{\Delta x}}\) и в итоге получить, что

  \begin{equation*}
    W = f''_{yx}(\xi_2, \zeta_2)
    \qquad
    (
      \xi_2 \in \interval{x_0}{x_0 + \Delta x},
      \zeta_2 \in \interval{y_0}{y_0 + \Delta y}
    )
  \end{equation*}

  В итоге, применяя предельный переход при \(\Delta x \to 0\) и \(\Delta y \to
  0\) имеем

  \begin{equation*}
    \begin{aligned}
      W = f''_{xy}(\xi_1, \zeta_1) = f''_{yx}(\xi_2, \zeta_2)
    \\
      \lim_{\rho \to 0} f''_{xy}(\xi_1, \zeta_1)
      = \lim_{\rho \to 0} f''_{xy}(\xi_2, \zeta_2)
      & \qquad
      (\rho = \sqrt{(\Delta x)^2 + (\Delta y)^2})
    \end{aligned}
  \end{equation*}

  В таком случае \(\xi_1 \to x_0\), \(\xi_2 \to x_0\), \(\zeta_1 \to y_0\) и
  \(\zeta_2 \to y_0\). По условию теоремы вторые производные непрерывны, значит
  \(f''_{xy}(x_0, y_0) = f''_{yx}(x_0, y_0)\).
\end{proof}

\begin{definition}
  Вторым дифференциалом функции двух переменных называется дифференциал
  дифференциала функции двух переменных.

  \begin{equation*}
    \dd^2 z  = \frac{\partial^2 z}{\partial x^2} \dd x^2
      + 2 \frac{\partial^2 z}{\partial x \partial y} \dd x \dd y
      + \frac{\partial^2 z}{\partial y^2} \dd y^2
  \end{equation*}
\end{definition}

\subheader{Вывод формулы для второго дифференциала}

По определению второй дифференциал равен \(\dd^2 z = \dd (\dd z)\). Подставим
первый дифференциал, получим

\begin{equation*}
  \dd^2 z = \dd \prh{\partder{z}{x} \dd x + \partder{z}{y} \dd y}
\end{equation*}

Еще раз применяем определение первого дифференциала.

\begin{enumerate}
\item
  Берем производную по \(x\) от того, что в скобках.

\item
  Умножаем её на \(\dd x\).

\item
  Берем производную по \(y\) от того, что в скобках.

\item
  Умножаем её на \(\dd y\).

\item
  Ответом будет сумма того, что получилось на шагах 2 и 4.
\end{enumerate}

Итого имеем

\begin{equation*}
  \begin{aligned}
    \dd^2 z
    = \prh{\partder{z}{x} \dd x + \partder{z}{y} \dd y}'_x \dd x
      + \prh{\partder{z}{x} \dd x + \partder{z}{y} \dd y}'_y \dd y
  \\
    \dd^2 z
    = \prh{
        \frac{\partial^2 z}{\partial x^2} \dd x
        + \frac{\partial^2 z}{\partial y \partial x} \dd y
      } \dd x
      + \prh{
        \frac{\partial^2 z}{\partial x \partial y} \dd x
        + \frac{\partial^2 z}{\partial y^2} \dd y
      } \dd y
  \\
    \dd^2 z = \frac{\partial^2 z}{\partial x^2} \dd x^2
      + 2 \frac{\partial^2 z}{\partial x \partial y} \dd x \dd y
      + \frac{\partial^2 z}{\partial y^2} \dd y^2
  \end{aligned}
\end{equation*}

\begin{theorem}
  Дифференциал второго (и более высокого) порядка функции двух переменных не
  сохраняет инвариантность формы в общем случае.
\end{theorem}

\begin{proof}
  Рассмотрим \(z = f(u, v)\), пусть \(u = u(t)\) и \(v = v(t)\). По определению
  второго дифференциала получаем

  \begin{equation*}
    \dd^2 z
    = \dd (\dd z)
    = \dd \prh{z'_u \dd u + z'_v \dd v}
    = \prh{z'_u \under{\dd u}{u'_t \dd t}
      + z'_v \under{\dd v}{v'_t \dd t}}'_t \dd t
    = \dd^2 z = (z'_u u'_t \dd t + z'_v v'_t \dd t)'_t \dd t
  \end{equation*}
  
  Вынесем \(\dd t\) за скобки и возьмём производную

  \begin{equation*}
    \dd^2 z = \prh[\Big]{
      (z'_u)'_t \cdot u'_t
      + z'_u u''_{tt}
      + (z'_v)'_t \cdot v'_t
      + z'_v v''_{tt}
    } \dd t^2
  \end{equation*}

  Рассмотрим \((z'_u)'_t\), это производная сложной функции нескольких двух
  переменных, поэтому \((z'_u)'_t = z''_{uu}u'_t + z''_{uv}v'_t\). Аналогично
  \((z'_v)'_t = z''_{vu} u'_t + z''_{vv} v'_t\). Подставляем это в формулу,
  получаем

  \begin{equation*}
    \begin{aligned}
      \dd^2 z = \prh{
        z''_{uu} (u'_t)^2
        + z''_{uv} v'_t u'_t
        + z'_u u''_{tt}
        + z''_{uv} u'_t v'_t
        + z''_{uu} (u'_t)^2
        + z'_v v''_{tt}
      } \dd t^2
    \\
      \dd^2 z = \prh[\bigg]{
        \prh[\Big]{
          z''_{uu} (u'_t)^2 + 2 \cdot z''_{uv} v'_t u'_t + z''_{uu} (u'_t)^2
        }
        + z'_u u''_{tt}
        + z'_v v''_{tt}
      } \dd t^2
    \end{aligned}
  \end{equation*}

  Скобка с тремя слагаемыми, умноженная на \(\dd t^2\), это дифференциал функции
  двух переменных второго порядка, а оставшиеся два слагаемых дают дифференциалы
  второго порядка функции одной переменной, значит \(\dd^2 z = \dd^2 z + z'_u
  \dd^2 u + z'_v \dd^2 v\). Противоречие, значит дифференциал второго порядка
  функции нескольких переменных не сохраняет инвариантность формы в общем
  случае.
\end{proof}

\begin{definition}
  Линейной параметризацией называется параметризация функции \(z = f(u, v)\) с
  помощью линейной замены \(u = a t + b\) и \(v = ct + d\).
\end{definition}

\begin{theorem}
  Дифференциал второго порядка функции двух переменных сохраняет инвариантность
  формы в случае линейной параметризации.
\end{theorem}

\begin{proof}
  В ходе доказательства неинвариантности формы второго дифференциала функции
  двух переменных было получено следующее равенство

  \begin{equation*}
    \dd^2 z = \dd^2 z + z'_u \dd^2 u + z'_v \dd^2 v 
  \end{equation*}

  Однако т.к. \(u\) и \(v\) это линейные функции, то их второй дифференциал
  будет равен нулю, значит

  \begin{equation*}
    \dd^2 z = \dd^2 z + z'_u \cdot 0 + z'_v \cdot 0 \implies \dd^2 z = \dd^2 z
  \end{equation*}

  Получили верное равенство, значит второй дифференциал сохраняет инвариантность
  формы при линейной параметризации.
\end{proof}
