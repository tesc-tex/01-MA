\subsection{%
  Предел последовательности. Свойства сходящихся последовательностей.%
}

\begin{definition}
  \(L \in \RR\) называется конечным пределом последовательности \(\seq{x_n}\),
  если

  \begin{equation*}
    \lim_{n \to \infty} x_n = L \iff
    \forall \epsilon > 0 \exists  n_0 \in \NN \given
    \forall n \ge n_0 \colon \abs{x_n - L} < \epsilon
  \end{equation*}
\end{definition}

\begin{definition}
  \(L = \pm \infty\) называется бесконечным пределом последовательности
  \(\seq{x_n}\), если

  \begin{equation*}
    \lim_{n \to \infty} x_n = L \iff
    \forall \epsilon > 0 \exists n_0 \in \NN \given
    \forall n \ge n_0 \colon \abs{x_n} > \epsilon
  \end{equation*}
\end{definition}
    
\begin{definition}
  Последовательность называется сходящейся (т.е. сходится), если она имеет
  конечный предел.
\end{definition}

\begin{definition}
  Подпоследовательность \(\seq{x_{n_k}}\) последовательности \(\seq{x_n}\) это
  последовательность полученная из \(\seq{x_n}\) удалением некоторых её членов
  без изменения их порядка.
\end{definition}

\begin{remark}
  Если последовательность имеет предел, то любая её подпоследовательность имеет
  такой же предел.

  \begin{equation*}
    \lim_{n \to \infty} x_n = L \implies \lim_{k \to \infty} x_{n_k} = L    
  \end{equation*}
\end{remark}

\begin{theorem}[Больцано--Вейерштрасса] \label{thr:lim-seq-subseq}
  Из любой ограниченной последовательности можно выделить сходящуюся
  подпоследовательность.
\end{theorem}

\begin{proof}
  Т.к. последовательность \(\seq{x_n}\) ограничена, то \(\exists a, b \given
  \forall n \in \NN \colon a \le x_n \le b\). Разделим отрезок \([a; b]\)
  пополам, тогда хотя бы одна из половин содержит бесконечно много элементов из
  последовательности \(\seq{x_n}\). Назовём эту половину \([a_1; b_1]\) и
  возьмём из неё произвольный элемент, который назовём \(x_{n_1}\).

  Продолжим выполнять эти действия, после \(k\)-ого шага получим, что \(x_{n_k}
  \in [a_k; b_k]\). Т.к. отрезки вложены друг в друга, то найдется единственное
  число \(c\), такое что \(\forall k \in \NN \colon c \in [a_k; b_k]\). Таким
  образом границы отрезков стремятся к числу \(c\), значит

  \begin{equation*} \label{eq:conv-sub-seq-1}
    \lim_{k \to \infty} a_k = \lim_{k \to \infty} b_k = c
  \end{equation*}

  Но при этом

  \begin{equation*} \label{eq:conv-sub-seq-2}
    x_{n_k} \in [a_k; b_k]
    \implies
    a_k \le x_{n_k} \le b_k
    \implies[предельный переход]
    \lim_{k \to \infty} a_k \le \lim_{k \to \infty} x_{n_k}
      \le \lim_{k \to \infty} b_k
  \end{equation*}

  Из \eqref{eq:conv-sub-seq-1} и \eqref{eq:conv-sub-seq-2} по теореме о двух
  милиционерах следует, что \(\lim_{k \to \infty} x_{n_k} = c\). Таким образом
  последовательность \(x_{n_k}\) имеет конечный предел, а значит она сходится.
\end{proof}

\begin{definition}
  Последовательность называется фундаментальной, если

  \begin{equation*}
    \forall \epsilon > 0 \exists n_0 > 0 \given
    \forall n > n_0, m > n_0 \colon \abs{x_n - x_m} < \epsilon
  \end{equation*}
\end{definition}

\begin{theorem}[Критерий Больцано--Коши для сходимости последовательностей]
  Последовательность фундаментальна \(\iff\) она сходится.
\end{theorem}

\begin{proof}
  \ness Распишем определение предела последовательности.

  \begin{equation*}
    \lim_{n \to \infty} x_n = L \iff
    \forall \epsilon > 0 \exists n_0 \given
    \forall n > n_0 \colon \abs{x_n - L} < \epsilon
  \end{equation*}

  Пусть \(\display{\epsilon_1 = \epsilon_2 = \frac{\epsilon}{2}}\), тогда
  
  \begin{equation*}
    \begin{aligned}
      \forall m > n_0 \colon \abs{x_m - L} < \epsilon_1
    \\
      \forall n > n_0 \colon \abs{x_n - L} < \epsilon_2
    \end{aligned}
  \end{equation*}

  Заметим, что

  \begin{equation*}
    \begin{aligned}
      \abs{x_m - x_n}
      = \abs{(x_m - L) + (L - x_n)}
      \le \abs{x_m - L} + \abs{L - x_n}
    \\
      \begin{rcases}
        \abs{x_m - L} < \epsilon_1 \\
        \abs{L - x_n} < \epsilon_2 \\
        \epsilon_1 + \epsilon_2 = \epsilon
      \end{rcases}
      \implies
      \abs{x_m - x_n } < \epsilon
    \end{aligned}
  \end{equation*}

  Таким образом последовательность \(\seq{x_n}\) фундаментальна по определению.

  \suff Покажем, что фундаментальная последовательность ограничена. Пусть
  \(\seq{x_n}\) фундаментальная последовательность, тогда

  \begin{equation*}
    \forall \epsilon > 0 \exists n_0 \given
    \forall n, m \ge n_0 \colon \abs{x_n - x_m} < \epsilon
  \end{equation*}

  Возьмём \(\epsilon = 1\), тогда

  \begin{equation*}
    \abs{x_n}
    = \abs{(x_n - x_{n_0}) + x_{n_0}}
    \le \abs{x_n - x_{n_0}} + \abs{x_{n_0}}
  \end{equation*}

  Т.к. \(\seq{x_n}\) фундаментальна и \(\epsilon = 1\), то первый модуль не
  превосходит единицу. Также заметим, что т.к. мы взяли фиксированный
  \(\epsilon\), то \(x_{n_0}\) это некая константа. Таким образом \(\forall n >
  n_0 \colon \abs{x_n} \le A\), где \(A\) это некая константа, которой мы
  обозначили всю правую часть. Мы видим, что последовательность \(\seq{x_n}\)
  ограничена.

  Последовательность \(\seq{x_n}\) ограниченна, значит по теореме
  \ref{thr:lim-seq-subseq} из неё можно выделить сходящуюся
  подпоследовательность \(\seq{x_{n_k}}\). Допустим эта подпоследовательность
  имеет предел равный \(L\), покажем, что и исходная последовательность тоже
  имеет предел \(L\).

  По определению фундаментальной последовательности \(\seq{x_n}\) имеем

  \begin{equation*}
    \forall \epsilon_1 > 0 \exists n_1 \given
    \forall n, m > n_1 \colon \abs{x_n - x_m} < \epsilon_1
  \end{equation*}

  По определению сходящейся последовательности \(\seq{x_{n_k}}\) имеем

  \begin{equation*}
    \forall \epsilon_2 > 0 \exists n_2 \given
    \forall n_k > n_2 \given \abs{x_{n_k} - L} < \epsilon_2
  \end{equation*}

  Пусть \(n_0 = \max(n_1, n_2)\), тогда оба условия выполнены. Обозначим \(m =
  n_k\) и \(\display{\epsilon_1 = \epsilon_2 = \frac{\epsilon}{2}}\), тогда

  \begin{equation*}
    \abs{x_n - L}
    = \abs{(x_n - x_m) + (x_m - L)}
    < \epsilon_1 + \epsilon_2
    = \epsilon
  \end{equation*}

  Итого имеем

  \begin{equation*}
    \forall \epsilon > 0 \exists n_0 \given
    \forall n > n_0 \colon \abs{x_n - L} < \epsilon
  \end{equation*}

  Значит последовательность \(\seq{x_n}\) имеет предел \(L\), а значит она
  сходится.
\end{proof}

\begin{theorem}[Вейерштрасса о сходимости последовательности]
  Если неубывающая (невозрастающая) последовательность ограничена, то она
  сходится.
\end{theorem}

\begin{proof}
  Если последовательность \(\seq{x_n}\) ограничена, то она имеет супремум, т.е.
  
  \begin{equation*}
    \exists S \given \forall \epsilon > 0 \exists n_0 \given
    S - \epsilon < x_{n_0} \le S
  \end{equation*}

  Т.к. последовательность неубывает, то \(n_0 < n \implies x_{n_0} \le x_n\).
  Объединяя эти две формулы, получаем

  \begin{equation*}
    \begin{aligned}
      \forall \epsilon > 0 \exists n_0 \given
      \forall n > n_0 \colon S - \epsilon < x_{n_0} \le x_n \le S
    \\
      \forall \epsilon > 0 \exists n_0 \given
      \forall n > n_0 \colon S - \epsilon < x_n < S + \epsilon
    \\
    \forall \epsilon > 0 \exists n_0 \given
    \forall n > n_0 \colon \abs{x_n - S} < \epsilon
    \end{aligned}
  \end{equation*}

  Таким образом последовательность \(\seq{x_n}\) имеет конечный предел по
  определению, а значит она сходится.
\end{proof}

\begin{theorem}
  Если последовательность сходится, то она ограничена.
\end{theorem}

\begin{proof}
  Если последовательность сходится, значит она имеет конечный предел \(\lim_{n
  \to \infty} x_n = L \in \RR\). Распишем предел последовательности по
  определению, после чего раскроем модуль.

  \begin{equation*}
    \begin{aligned}
      \lim_{n \to \infty} x_n = L \iff
      \forall \epsilon > 0 \exists n_0 \in \NN \given
      \forall n > n_0 \colon \abs{x_n - L} < \epsilon
    \\
      -\epsilon + L < x_n < L + \epsilon
    \end{aligned}   
  \end{equation*}
  
  Таким образом мы видим, что последовательность ограничена.
\end{proof}
