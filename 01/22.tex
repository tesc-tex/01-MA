\subsection{%
  Производные высших порядков. Дифференциал \(2\)-го порядка.%
}

\begin{definition}
  Если производная функции \(f(x)\) в точке \(x_0\) дифференцируема в этой
  точке, то говорят, что определена вторая производная \(f''(x) = (f'(x))'\).
\end{definition}

Аналогичны определены производные и более высоких порядков \(f'''(x),
f^{(IV)}(x), f^{(V)}(x) \dotsc\)

Данное определение позволяет выделить класс бесконечно дифференцируемых функций
в точке, например \(\sin x, e^x\) и т.д. Полиномы степени \(n\) называют \(n\)
раз дифференцируемыми (т.к. после \(n\) дифференцирований производная будет
оставаться равной \(0\)).

\begin{definition}
  Дифференциалом функции второго порядка называется дифференциал дифференциала
  функции.

  \begin{equation*}
    \dd^2 y = \dd (\dd y) = y''(x) \dd x^2
  \end{equation*}
\end{definition}

\begin{remark}
  Дифференциал \(n\)--ого порядка определен как \(\dd^n y = y^{(n)} \dd x^n\).
\end{remark}

\begin{theorem}
  Дифференциал второго (и более высокого) порядка не сохраняет инвариантность
  формы.
\end{theorem}

\begin{proof}
  Рассмотрим \(y = f(x)\), пусть \(x = u(t)\). По определению второго
  дифференциала имеем

  \begin{equation*}
    \dd^2 y
    = \dd (\dd y)
    = \dd (y'_t \dd t)
  \end{equation*}

  Пусть \(y'_t \dd t = \bigstar\), применим определение первого дифференциала.

  \begin{equation*}
    \dd (\bigstar)
    = (\bigstar)' \dd t
    = (y'_t \dd t)' \dd t
  \end{equation*}

  Раскроем полученное выражение по правилу производной произведения.

  \begin{equation*}
    (y''_t \dd t + y'_t (\dd t)'_t) \dd t
    = y''_t \under{\dd t \dd t}{\dd t^2}
      + y'_t \under{(\dd t)'_t \dd t}{\dd (\dd t) = \dd^2 t}
    = y''_t \dd t^2 + y'_t \dd^2 t
  \end{equation*}
     
  При этом по определению второго дифференциала \(\dd^2 y = y''_x \dd x^2\).
  Таким образом дифференциал второго порядка не сохраняет инвариантность формы.
\end{proof}
