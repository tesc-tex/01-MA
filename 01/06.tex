\subsection{%
  Бесконечно малые, бесконечно большие функции. Свойства.%
}

\begin{definition}
  Функция \(\infsmall(x)\) называется бесконечно малой (б.м.) в точке \(a\),
  если её предел в этой точке равен нулю.
\end{definition}

\begin{definition}
  Функция \(\infbig(x)\) называется бесконечно большой (б.б.) в точке \(a\),
  если её предел в этой точке равен бесконечности.
\end{definition}

\begin{remark}
  Функция не бывает бесконечно малой или бесконечно большой где угодно, только в
  определённой точке. Например, \(f(x) = \frac{1}{x}\), тогда

  \begin{enumerate}
  \item
    \(f(x)\)~--- б.б. в точке \(x = 0\).

  \item
    \(f(x)\)~--- б.м. в точке \(x = +\infty\).
  \end{enumerate}  
\end{remark}

\begin{theorem} \label{thr:sum-inf-a}
  Сумма бесконечно малых в точке равна бесконечно малой в этой же точке.

  \begin{equation*}
    \infsmall_1(x) + \infsmall_2(x) = \infsmall(x)
  \end{equation*}
\end{theorem}

\begin{proof}
  Распишем бесконечно малые по определению.  

  \begin{equation*}
    \begin{aligned}
      \lim_{x \to a} \infsmall_1(x) = 0 \iff
      \forall \epsilon_1 > 0 \exists \delta_1 > 0 \given
      \forall x \in \nearo{a}{\delta_1} \cap E \implies
      \abs{\infsmall_1(x)} < \epsilon_1
    \\
      \lim_{x \to a} \infsmall_2(x) = 0 \iff
      \forall \epsilon_2 > 0 \exists \delta_2 > 0 \given
      \forall x \in \nearo{a}{\delta_2} \cap E \implies
      \abs{\infsmall_2(x)} < \epsilon_2
    \end{aligned}
  \end{equation*}

  Пусть \(\delta = \min(\delta_1, \delta_2)\), тогда по свойству окрестностей
  оба условия будет выполняться. Также возьмём \(\epsilon > 0\), такую что
  \(\epsilon_1 = \epsilon_2 =\frac{\epsilon}{2}\). Т.к. \(\epsilon\) проходит
  все вещественные значения, то и \(\epsilon_1\) и \(\epsilon_2\) будут
  проходить все вещественные значения, значит условия в пределе все еще будут
  истинными, получим

  \begin{equation*}
    \begin{aligned}
      \lim_{x \to a} \infsmall_1(x) = 0 \iff
      \forall \epsilon > 0 \exists \delta > 0 \given
      \forall x \in \nearo{a}{\delta} \cap E \implies
      \abs{\infsmall_1(x)} < \frac{\epsilon}{2}
    \\
      \lim_{x \to a} \infsmall_2(x) = 0 \iff
      \forall \epsilon > 0 \exists \delta > 0 \given
      \forall x \in \nearo{a}{\delta} \cap E \implies
      \abs{\infsmall_2(x)} < \frac{\epsilon}{2}
    \end{aligned}
  \end{equation*}

  Сложим полученные два неравенства и воспользуемся свойством модуля суммы.

  \begin{equation*} \label{eq:sum-inf-a} \tag{MOD}
    \begin{rcases}
      \abs{\infsmall_1(x)} + \abs{\infsmall_2(x)} < \epsilon \\
      \abs{\infsmall_1(x) + \infsmall_2(x)} \le
        \abs{\infsmall_1(x)} + \abs{\infsmall_2(x)}
    \end{rcases}
    \implies
    \abs{\infsmall_1(x) + \infsmall_2(x)} < \epsilon
  \end{equation*}

  Итого

  \begin{equation*}
    \forall \epsilon > 0 \exists \delta > 0 \given
    \forall x \in \nearo{a}{\delta} \cap E \implies
    \abs{\infsmall_1(x) + \infsmall_2(x)} < \epsilon
  \end{equation*}

  Это и есть определение предела, т.е. \(\lim_{x \to a} (\infsmall_1(x) +
  \infsmall_2(x)) = 0\).
\end{proof}

\begin{theorem}
  Сумма бесконечно больших \textbf{одного знака} в точке равна бесконечно
  большой в этой же точке.

  \begin{equation*}
    \infbig_1(x) + \infbig_2(x) = \infbig(x)
    \qquad
    (\infbig_1(x) \cdot \infbig_2(x) > 0)
  \end{equation*}
\end{theorem}

\begin{proof}
  Доказательство аналогично доказательству про сумму б.м. (\ref{thr:sum-inf-a}).
  При раскрытии модуля \eqref{eq:sum-inf-a} поможет тот факт, что б.б. одного
  знака.

  \begin{equation*}
    \abs{\infbig_1(x)} + \abs{\infbig_2(x)} > \epsilon
    \implies
    \abs{\infbig_1(x) + \infbig_2(x)} > \epsilon
  \end{equation*}
\end{proof}

\begin{theorem}
  Произведение бесконечно малых в точке равно бесконечно малой в этой же точке.
  
  \begin{equation*}
    \infsmall_1(x) \cdot \infsmall_2(x) = \infsmall(x)
  \end{equation*}
\end{theorem}

\begin{proof}
  Доказательство аналогично доказательству про сумму б.м. (\ref{thr:sum-inf-a}).
  Единственное отличие в том, что нужно положить
  \(\epsilon_1 = \epsilon_2 = \sqrt{\epsilon}\).
\end{proof}

\begin{theorem}
  Произведение бесконечно больших в точке равно бесконечно большой в этой же
  точке.

  \begin{equation*}
    \infbig_1(x) \cdot \infbig_2(x) = \infbig(x)
  \end{equation*}
\end{theorem}

\begin{proof}
  Доказательство аналогично доказательству про произведение б.м.
\end{proof}

\begin{theorem}
  Произведение бесконечно малой в точке на ограниченную функцию равно бесконечно
  малой в этой же точке.

  \begin{equation*}
    \infsmall_1(x) \cdot f(x) = \infsmall(x)
  \end{equation*}
\end{theorem}

\begin{proof}
  Доказательство аналогично доказательству про сумму б.м. (\ref{thr:sum-inf-a}).
  Единственное отличие в том, что нужно полагать \(\display{e_1 =
  \frac{e}{M}}\), где \(M\) это граница функции \(f(x)\).
\end{proof}

\begin{theorem}
  Произведение бесконечно большой в точке на ограниченную функцию (не равную
  нулю) равно бесконечно большой в этой же точке.

  \begin{equation*}
    \infbig_1(x) \cdot f(x) = \infbig(x)
    \qquad   
    (f(a) \neq 0)
  \end{equation*}
\end{theorem}

\begin{proof}
  Доказательство аналогично доказательству про произведение б.м. на ограниченную
  функцию.
\end{proof}
