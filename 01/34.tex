\subsection{%
  Полный дифференциал функции двух переменных. Инвариантность формы.%
}

\begin{definition}
  Частным дифференциалом функции двух переменных называется произведение
  соответствующей частной производной на частное приращение. Например, частный
  дифференциал по \(x\) имеет вид \(\display{\dd_x z =\partder{z}{x} \Delta
  x}\).
\end{definition}

\begin{definition}
  Функция двух переменных \(z = f(x, y)\) называется дифференцируемой в точке
  \(M_0 (x_0, y_0)\), если ей полное приращение представимо в виде

  \begin{equation*}
    \Delta z = A \Delta x + B \Delta y + \smallo(\rho)
    \qquad
    \prh{\rho = \sqrt{(\Delta x)^2 + (\Delta y)^2}}
  \end{equation*}
\end{definition}

\begin{definition}
  Функция двух переменных называется дифференцируемой на множестве \(D\), если
  она дифференцируема в каждой точке этого множества.
\end{definition}

\begin{theorem}
  Если функция \(z = f(x, y)\) дифференцируема в точке \(M_0 (x_0, y_0)\), то у
  нее существуют частные производные \(z'_x\) и \(z'_y\) в этой точке.
\end{theorem}

\begin{proof}
  Т.к. \(f(x, y)\) дифференцируема в точке \(M_0 (x_0, y_0)\), то её полное
  приращение представимо в виде

  \begin{equation*}
    \Delta z = A \Delta x + B \Delta y + \smallo(\rho)
    \qquad
    \prh{\rho = \sqrt{(\Delta x)^2 + (\Delta y)^2}}
  \end{equation*}

  Зафиксируем \(y = const\), тогда \(\Delta y = 0\), получим \(\Delta z = A
  \Delta x + \smallo(\Delta x)\). Из этого по критерию дифференцируемости
  функции одной переменной следует, что существует производная \(z_x' = A\).
  Аналогично рассматривая \(y\), получим что существует \(z'_y = B\).
\end{proof}

\begin{corollary}
  Таким образом мы доказали, что дифференцируемая функция представима в виде

  \begin{equation*}
    \Delta z = \partder{z}{x} \Delta x + \partder{z}{y} \Delta y
      + \smallo(\rho)
    \qquad
    \prh{\rho = \sqrt{(\Delta x)^2 + (\Delta y)^2}}
  \end{equation*}
\end{corollary}

\begin{definition}
  Полным дифференциалом функции двух переменных называется линейная относительно
  \(\Delta x\) и \(\Delta y\) часть её полного приращения.

  \begin{equation*}
    \dd z = \partder{z}{x} \dd x + \partder{z}{y} \dd y
  \end{equation*}
\end{definition}

\begin{remark}
  Полный дифференциал равен сумме частных дифференциалов.
\end{remark}

\begin{theorem}[Достаточное условие дифференцируемости функции двух переменных]
  Если функция \(z = f(x, y)\) в точке \(M_0\) имеет непрерывные частные
  производные \(z'_x\) и \(z'_y\), то она дифференцируема в этой точке.
\end{theorem}

\begin{proof}
  Рассмотрим полное приращение функции \(z = f(x, y)\).

  \begin{equation*}
    \begin{aligned}
      \Delta z = z (x + \Delta x, y + \Delta y) - z(x, y)
    \\
      \Delta z = \prh[\Big]{z(x + \Delta x, y + \Delta y) - z(x, y + \Delta y)}
        + \prh[\Big]{z(x, y + \Delta y) - z(x, y)}
    \end{aligned}
  \end{equation*}

  В каждой из скобок мы можем рассматривать функцию \(z(x, y)\) как функцию
  одной переменной в плоскостях \(y + \Delta y\) и \(x\) соответственно.
  Применим формулу Лагранжа к каждой из скобок, получим

  \begin{equation*}
    \Delta z = z'_x(\xi, y + \Delta y) \Delta x + z'_y(x, \mu) \Delta y
    \qquad
    \begin{cases}
      \xi \in (x, x + \Delta x) \\
      \mu \in (y, y + \Delta y)
    \end{cases}
  \end{equation*}

  Введем следующие обозначения

  \begin{equation*}
    \rho = \sqrt{(\Delta x)^2 + (\Delta y)^2}
    \qquad
    \rho \to 0 \implies \begin{cases}
      x \to 0 \\
      y \to 0
    \end{cases}
  \end{equation*}

  Т.к. по условию теоремы частные производные непрерывны, то

  \begin{equation*}
    \begin{aligned}
      \begin{rcases}
        \xi \in (x, x + \Delta x) \\
        \Delta x \to 0
      \end{rcases}
      & \implies
      \lim_{\rho \to 0} z'_x(\xi, y + \Delta y) = z'_x(x, y)
    \\
      \begin{rcases}
        \mu \in (y, y + \Delta y) \\
        \Delta y \to 0
      \end{rcases}
      & \implies
      \lim_{\rho \to 0} z'_y(x, \mu) = z'_y(x, y)
    \end{aligned}
  \end{equation*}

  Воспользуемся представлением функции пределом.

  \begin{equation*}
    \begin{aligned}
      z'_x(\xi, y + \Delta y) = z'_x(x, y) + \alpha(\rho)
    \\
      z'_y(x, \mu) = z'_y(x, y) + \beta(\rho)
    \end{aligned}
  \end{equation*}
  
  Значит полное приращение функции \(f(x, y)\) можно представить в виде

  \begin{equation*}
    \begin{aligned}
      \Delta z = (z'_x(x, y) + \alpha(\rho)) \Delta x
        + (z'_y(x, y) + \beta(\rho)) \Delta y
    \\
      \Delta z = z'_x(x, y) \Delta x + z'_y(x, y) \Delta y
      + \alpha(\rho) \Delta x + \beta(\rho) \Delta y
    \end{aligned}
  \end{equation*}

  Докажем, что \(\alpha(\rho) \Delta x + \beta(\rho) \Delta y = \smallo(\rho)\).
  Для этого рассмотрим предел

  \begin{equation*}
    \lim_{\rho \to 0} \frac{\alpha(\rho) \Delta x
      + \beta(\rho) \Delta y}{\rho}
    = \lim_{\rho \to 0} \prh{\alpha(\rho) \frac{\Delta x}{\rho}} + 
      \lim_{\rho \to 0} \prh{\beta(\rho) \frac{\Delta y}{\rho}}
  \end{equation*}

  Заметим, что \(\display{\sqrt{\frac{(\Delta x)^2}{\rho^2} +
  \frac{(\Delta y)^2}{\rho^2}} = 1}\). Из этого следует, что функции
  \(\display{\frac{\Delta x}{\rho}}\) и \(\display{\frac{\Delta y}{\rho}}\)
  ограничены (если быть точным, то каждая из них не превышает единицу). По
  свойству бесконечно малых получаем

  \begin{equation*}
    \lim_{\rho \to 0} \prh{\alpha(\rho) \frac{\Delta x}{\rho}} + 
      \lim_{\rho \to 0} \prh{\beta(\rho) \frac{\Delta y}{\rho}}
    = \lim_{\rho \to 0} \alpha(\rho) + \lim_{\rho \to 0} \beta(\rho)
    = 0
  \end{equation*}
  
  Значит \(\alpha(\rho) \Delta x + \beta(\rho) \Delta y = \smallo(\rho)\) по
  определению бесконечно малой функции более высокого порядка. В итоге получаем,
  что полное приращение функции \(z = f(x, y)\) представимо в  виде \(\Delta z =
  z'_x \Delta x + z'_y \Delta y + \smallo(\rho)\). Таким образом функция \(z =
  f(x, y)\) дифференцируема по определению.
\end{proof}

\begin{theorem}
  Первый полный дифференциал сохраняет инвариантность формы.
\end{theorem}

\begin{proof}
  Рассмотрим \(z = f(u, v)\), пусть \(u = u(x, y)\) и \(v = v(x, y)\). По
  определению полного дифференциала получаем 

  \begin{equation*}
    \begin{aligned}
      \dd z = \partder{z}{x} \dd x + \partder{z}{y} \dd y
    \\
      \dd z = \prh{\partder{z}{u} \cdot \partder{u}{x}
        + \partder{z}{v} \cdot \partder{v}{x}} \dd x
        + \prh{\partder{z}{u} \cdot \partder{u}{y}
        + \partder{z}{v} \cdot \partder{v}{y}} \dd y
    \\
      \dd z = \partder{z}{u} \prh{\partder{u}{x} \dd x + \partder{u}{y} \dd y}
        + \partder{z}{v} \prh{\partder{v}{x} \dd x + \partder{v}{y} \dd y}
    \\
      \dd z = \partder{z}{u} \dd u + \partder{z}{v} \dd v
    \end{aligned}
  \end{equation*}
  
  Таким образом, форма дифференциала не зависит от того, является аргумент
  функции независимой переменной или функцией другого аргумента.
\end{proof}

\begin{remark}
  С помощью дифференциала функции двух переменных можно искать дифференциал
  функции одной переменной, заданной неявно.
\end{remark}

\begin{example}
  Пусть \(F(x, y(x)) = 0\). Рассмотрим её как функцию двух независимых
  переменных и вычислим дифференциал.

  \begin{equation*}
    \dd F = \partder{F}{x} \dd x + \partder{F}{y} \dd y
  \end{equation*}

  Этот дифференциал будет равен нулю (т.к. исходная функция равна нулю), значит

  \begin{equation*}
    \partder{F}{x} + \partder{F}{y} \cdot \fullder{y}{x} = 0
    \implies
    \fullder{y}{x}
    = -\frac{\partder{F}{x}}{\partder{F}{y}}
    = -\frac{F'_x}{F'_y}
  \end{equation*}
\end{example}
