\subsection{%
  Определения непрерывной функции. Свойства функции, непрерывной на отрезке.%
} \label{sec:01-12}

Определения непрерывной функции были даны ранее (\ref{sec:01-11}).

\begin{theorem}[Вейерштрасса I]
  Функция, непрерывная на отрезке, ограничена на нём.  
\end{theorem}

\begin{proof}
  От противного: пусть \(f(x)\) неограниченна на отрезке \(\segment{a}{b}\).
  Тогда \(\forall A \exists x_A \in \segment{a}{b} \colon \abs{f(x_A)} > A\).
  Составим из \(x_A\) последовательность, которая будет лежать на отрезке
  \(\segment{a}{b}\), т.е. эта последовательность ограничена.
  
  По теореме Больцано--Вейерштрасса (\ref{thr:lim-seq-subseq}) из неё можно
  выделить сходящую к точке \(c \in \segment{a}{b}\) подпоследовательность, т.е.
  \(\lim_{k \to \infty} x_{n_k} = c\). Тогда, т.к. функция \(f(x)\) непрерывна
  на отрезке \(\segment{a}{b}\), значит она непрерывна и в точке \(c\), значит
  по определению непрерывности (по Гейне) \(\lim_{k \to \infty} f(x_{n_k}) =
  f(c)\), следовательно предел \textbf{конечный}, т.к. \(f(c) \in \RR\).

  С другой стороны, т.к. функция неограниченна, то \(\forall n_k \exists x_{n_k}
  \given f(x_{n_k}) > n_k\).  Т.к. \(\seq{x_{n_k}}\) это подпоследовательность
  последовательности \(\seq{x_n}\), то элемент в последовательности
  \(\seq{x_n}\) будет иметь номер не меньше, чем в подпоследовательности
  \(\seq{x_{n_k}}\), т.е. \(n_k \ge k\).
  
  Таким образом, объединяя полученные два неравенства получаем, что \(f(x_{n_k})
  > n_k \ge k\). Тогда по определению предела (по Гейне) \(\lim_{k \to \infty}
  f(x_{n_k}) = \infty\). Противоречие, т.к. функция должна иметь не более одного
  предела в точке (\ref{thr:lim-unique}).
\end{proof}

\begin{theorem}[Вейерштрасса II]
  Функция, непрерывная на отрезке, принимает на нём наибольшее и наименьшее
  значения.  
\end{theorem}

\begin{proof}
  Из первой теоремы Вейерштрасса следует, что т.к. функция \(f(x)\) непрерывна
  на отрезке, то она ограничена на этом отрезке, т.е. на отрезке
  \(\segment{a}{b}\) существует супремум \(M\). По определению супремума

  \begin{equation*}
    \begin{cases}
      \forall x \in \segment{a}{b} \given f(x) \le M \\
      \forall \epsilon > 0 \exists x_{\epsilon} \in \segment{a}{b} \given
        f(x_{\epsilon}) > M - \epsilon
    \end{cases}
  \end{equation*}
  
  Пусть \(\display{\epsilon = 1, \frac{1}{2}, \frac{1}{3}, \dotsc}\), тогда
  получим последовательность \(\seq{x_n}\), такую что \(\display{M - \frac{1}{n}
  < f(x_n) \le M}\).
  
  Значит по теореме о двух жандармах (\ref{thr:squeezed-func})
  \(\display{\lim_{n \to \infty} f(x_n) = M}\). Из ограниченной (т.к. \(x_n\)
  имеет конечный предел равный \(M\)) последовательности \(\seq{x_n}\) по
  теореме Больцано--Вейерштрасса (\ref{thr:lim-seq-subseq}) выделим сходящуюся к
  точке \(c_1 \in \segment{a}{b}\) подпоследовательность \(\seq{x_{n_k}}\).
  
  Т.к. функция непрерывна в точке \(c_1\), то по определению непрерывности (по
  Гейне) получаем \(\display{\lim_{k \to \infty} f(x_{n_k}) = f(c_1)}\). Но т.к.
  \(\seq{f(x_{n_k})}\) подпоследовательность последовательности
  \(\seq{f(x_n)}\), которая сходится к \(M\), то и \(\seq{f(x_{n_k})}\) также
  сходится к \(M\), таким образом \(\display{\lim_{k \to \infty} f(x_{n_k}) =
  M}\).
  
  Т.к. предел в точке единственный (\ref{thr:lim-unique}), то получаем, что
  \(f(c_1) = M\). Таким образом, мы показали, что функция \(f(x)\) принимает на
  отрезке \(\segment{a}{b}\) наибольшее значение равное \(f(c_1) = M\).
  Аналогично можно показать, что \(f(x)\) принимает наименьшее значение на
  отрезке \(\segment{a}{b}\).
\end{proof}

\begin{theorem}[Больцано--Коши I]
  \begin{equation*}
    \begin{rcases}
      f(x) \iscont{\segment{a}{b}} \\
      f(a) f(b) < 0
    \end{rcases}
    \implies
    \exists \xi \in \interval{a}{b} \given f(\xi) = 0    
  \end{equation*}
\end{theorem}

\begin{proof}
  Разделим исходный отрезок пополам точкой \(x_0\), если \(f(x_0) = 0\), то
  искомая точка найдена. В противном случае, если сдвинем одну из границ отрезка
  \(\segment{a}{b}\) в точку \(x_0\) так, чтобы исходное условие теоремы
  выполнялось

  \begin{enumerate}
  \item
    \(f(x_0) < 0, f(a) < 0 \implies a \to x_0\)

  \item 
    \(f(x_0) < 0, f(b) < 0 \implies b \to x_0\)

  \item 
    \(f(x_0) > 0, f(a) > 0 \implies a \to x_0\)

  \item
    \(f(x_0) > 0, f(b) > 0 \implies b \to x_0\)
  \end{enumerate}
  
  Обозначим новый отрезок \(\segment{a_1}{b_1}\), снова разделим его пополам и
  т.д. В итоге мы либо найдем искомую точку за конечное число разбиений, либо
  получим последовательность отрезков \(\segment{a_n}{b_n}\), таких что \(f(a_n)
  < 0 < f(b_n)\). Воспользуемся предельным переходом

  \begin{equation*} \label{eq:B-C-1} \tag{1}
    \lim_{n \to \infty} f(a_n) \le 0 \le \lim_{n \to \infty} f(b_n)
  \end{equation*}

  Т.к. отрезки вложены друг в друга, то возьмём их общую точку \(\mu\). Тогда,
  если длина отрезков стремится к нулю, то их концы стремятся к точке \(\mu\), а
  учитывая непрерывность функции получаем

  \begin{equation*} \label{eq:B-C-2} \tag{2}
    \begin{aligned}
      \lim_{n \to \infty} a_n = \lim_{n \to \infty} b_n = \mu
    \\
      \lim_{n \to \infty} f(a_n) = \lim_{n \to \infty} f(b_n) = f(\mu)
    \end{aligned}
  \end{equation*}

  Подставим \eqref{eq:B-C-2} в \eqref{eq:B-C-1}, получим что \(f(\mu) = 0\) и
  искомая точка найдена.
\end{proof}

\begin{theorem}[Больцано--Коши II]
  Если \(f(x)\) непрерывна на отрезке \(\segment{a}{b}\) и \(m\)~--- наименьшее
  значение функции на этом отрезке, а \(M\)~--- наибольшее, то она принимает все
  значения из отрезка \(\segment{m}{M}\).
\end{theorem}

\begin{proof}
  Пусть минимум на отрезке \(\segment{a}{b}\) достигается в точке \(c_1\), а
  максимум достигается в точке \(c_2\), т.е. \(f(c_1) = m, f(c_2) = M\). Причем
  понятно, что \(\segment{c_1}{c_2} \subseteq \segment{a}{b}\). Возьмём
  произвольное значение \(С\) из отрезка \(\interval{m}{M}\) и рассмотрим
  вспомогательную функцию \(h(x) = f(x) - C\), которая непрерывна как разность
  непрерывных функций. Тогда

  \begin{equation*}
    \begin{aligned}
      h(c_1) = m - C \qquad h(c_2) = M - C
    \\
      C \in \interval{m}{M} \implies m - C < 0, M - C > 0
    \end{aligned}  
  \end{equation*}
  
  Применим первую теорему Больцано--Коши для функции \(h(x)\) на отрезке
  \(\segment{c_1}{c_2}\) (мы уже показали, что функция \(h(x)\) удовлетворяет
  условиям этой теоремы). Получим, что

  \begin{equation*}
    \begin{aligned}
      \exists \xi \in \segment{c_1}{c_2} \given h(\xi) = 0
    \\
      \begin{rcases}
        \segment{c_1}{c_2} \subseteq \segment{a}{b} \\
        h(\xi) = f(\xi) - C = 0
      \end{rcases}
      \implies
      C = f(\xi)
    \end{aligned}
  \end{equation*}

  Значит для произвольного \(C \in \interval{m}{M}\) мы нашли такую точку
  \(\xi\), что \(f(\xi) = С\), при этом значения функция равные \(m\) и \(M\)
  достигаются в точках \(c_1\) и \(c_2\). Таким образом для любого числа \(C\)
  из отрезка \(\segment{m}{M}\) найдется точка \(\xi\) такая, что \(f(\xi) =
  C\).
\end{proof}
