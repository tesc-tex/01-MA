\subsection{%
  Определение функции двух переменных. Предел и непрерывность функции.%
}

\begin{definition}
  Функцией двух переменных называется отображение \(f \colon \RR^2 \to \RR\),
  такое что

  \begin{equation*}
    \forall M(x, y) \in D \given \existsone N(x, y, z(x, y)) \given z = f(M)
  \end{equation*}
\end{definition}

\begin{definition}
  Областью определения функции двух переменных будет некоторая область \(D\
  \subseteq \RR^2\).
\end{definition}

\begin{definition}
  Окрестностью точки \(M_0 (x_0, y_0)\) радиуса \(\delta\) будет круг, такой что

  \begin{equation*}
    \near{M_0}{\delta} = \set{(x, y) \given \sqrt{(x - x_0)^2 + (y - y_0)^2} <
      \delta}
  \end{equation*}
\end{definition}

\begin{remark}
  В дальнейшем данный корень будем обозначать \(\rho = \sqrt{(x - x_0)^2 + (y -
  y_0)^2}\).
\end{remark}

\begin{definition}
  Проколотой окрестностью точки \(M_0 (x_0, y_0)\) радиуса \(\delta\) будет
  называться окрестность точки \(M_0\) без точки \(M_0\).

  \begin{equation*}
    \nearo{M_0}{\delta} = \near{M_0}{\delta} \setminus \set{M_0}
  \end{equation*}
\end{definition}

\begin{definition}
  Точка \(M\) называется точкой сгущения множества \(D\), если пересечение любой
  её проколотой \(\delta\)--окрестности с этим множеством не пусто.
\end{definition}

\begin{definition}
  Область называется замкнутой, если она содержит все свои (конечные) точки
  сгущения (\figref{01_31_01}). Область называется открытой, если она не
  замкнута (\figref{01_31_02}).
\end{definition}

\gallerydouble
  {01_31_01}{Замкнутая область}
  {01_31_02}{Открытая область}

\begin{definition}
  Точка сгущения называется граничной точкой множества, если любая её
  \(\delta\)--окрестность не полностью лежит в этом множестве.
\end{definition}

\begin{definition}
  Множество граничных точек множества называется границей множества
  \(\Gamma_D\).
\end{definition}

\begin{definition}
  Предел функции двух переменных (в общем смысле) это

  \begin{equation*}
    \lim_{M \to M_0} z(M) = L \iff
    \forall \epsilon > 0 \exists \delta > 0 \given
    \forall M \in \nearo{M_0}{\delta} \cap D \implies
    z(M) \in \near{L}{\epsilon}
  \end{equation*}
\end{definition}

\begin{remark}
  Т.к. на плоскости бесконечно много способов приблизиться к точке, то в
  некоторых направления предел может существовать, а в некоторых - нет. Предел в
  общем смысле говорит о том, что при приближении к точке \textbf{с любой
  стороны} значение функции будет приближаться к \(L\).
\end{remark}

\begin{example}
  Рассмотрим функцию \(\display{z = \frac{x^2 - y^2}{x^2 + y^2}}\) в точке \(M_0
  (0, 0)\).

  \begin{equation*}
    \begin{aligned}
      \text{Путь } l_1 \colon \qquad &
      \begin{rcases}
        x = t \\
        y = t
      \end{rcases}
      \implies
      \lim_{l_1 \colon M \to M_0} \frac{t^2 - t^2}{t^2 + t^2} = 0
    \\
      \text{Путь } l_2 \colon \qquad &
      \begin{rcases}
        x = t \\
        y = 0
      \end{rcases}
      \implies
      \lim_{l_2 \colon M \to M_0} \frac{t^2}{t^2} = 1
    \end{aligned}
  \end{equation*}

  Таким образом мы видим, что при приближении к точке \(M_0\) по разным
  направлениям получаются разные пределы, значит предела в общем смысле не
  существует.
\end{example}

\begin{definition}
  Если сначала взять предел в сечении \(x = const\) рассматривая \(z(x, y)\) как
  функцию одной переменной, а потом взять предел по \(y\), то получится
  повторный предел, который обозначается \textbf{(порядок важен!)} следующим
  образом

  \begin{equation*}
    L_{xy} = \lim_{y \to y_0} \lim_{x \to x_0} z(x, y)
  \end{equation*}

  Если же сначала взять предел по \(y\), а потом по \(x\), то получится
  \textbf{другой} повторный предел.

  \begin{equation*}
    L_{yx} = \lim_{x \to x_0} \lim_{y \to y_0} z(x, y)
  \end{equation*}

  Если один из пределов в составе повторного не существует, то и сам повторный
  предел не существует.
\end{definition}

\begin{remark}[О порядке переменных в повторных пределах]
  Снова рассмотрим функцию \(\display{z = \frac{x^2 - y^2}{x^2 + y^2}}\) в точке
  \(M_0 (0, 0)\). Имеем

  \begin{equation*}
    \begin{aligned}
      \lim_{x \to 0} \lim_{y \to 0} \frac{x^2 - y^2}{x^2 + y^2}
      & = \lim_{x \to 0} 1
      = 1
    \\
      \lim_{y \to 0} \lim_{x \to 0} \frac{x^2 - y^2}{x^2 + y^2}
      & = \lim_{y \to 0} (-1)
      = -1      
    \end{aligned}
  \end{equation*}

  Таким образом порядок переменных в повторных пределах важен.
\end{remark}

\begin{remark}
  Существование повторных пределов никак не связано с существованием предела в
  общем смысле. Предел в общем смысле может существовать, а повторные
  пределы~--- нет или наоборот.
\end{remark}

\begin{definition}
  \begin{equation*}
    z(x, y) \iscontd{M_0} \bydef \lim_{M \to M_0} z(x, y) = z(M_0)
  \end{equation*}
\end{definition}

\begin{definition}
  Функция нескольких переменных непрерывна в области \(D\), если она непрерывна
  в каждой точке этой области.
\end{definition}

\begin{remark}
  Все свойства непрерывных функций одной переменных, описанные ранее
  (\ref{sec:01-12}), справедливы и для непрерывной функции нескольких
  переменных.

  \begin{enumerate}
  \item
    Функция, непрерывная в области, ограничена на ней.

  \item
    Функция, непрерывная в области, принимает на ней наибольшее и наименьшее
    значения.

  \item
    Если функция, непрерывная в области, принимает на ней значения разных
    знаков, то найдется точка, принадлежащая этой области, где значение функции
    равно нулю.

  \item
    Функция, непрерывная в области, принимает на ней все значения от наименьшего
    до наибольшего.
  \end{enumerate}
\end{remark}
