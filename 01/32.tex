\subsection{%
  Частные производные функции двух переменных.%
}

\begin{definition}
  Частным приращением функции двух переменных называется

  \begin{enumerate}
  \item
    по переменной \(x \given \Delta_x z = z(x + \Delta x, y) - z(x, y)\)

  \item
    по переменной \(y \given \Delta_y z = z(x, y + \Delta y) - z(x, y)\)
  \end{enumerate}

  Т.е. одна из переменных изменяется, а вторая остается константной.
\end{definition}

\begin{definition}
  Полным приращение функции двух переменных называется \(\Delta z = z(x + \Delta
  x, y + \Delta y) - z(x, y)\).
\end{definition}

\begin{remark}
  Полное приращение \textbf{НЕ равно} сумме частных приращений.
\end{remark}

\begin{definition}
  Частной производной функции двух переменных называют

  \begin{enumerate}
  \item
    по переменной \(x \given
      \display{\frac{\partial z}{\partial x}
      = z'_x
      = \lim_{\Delta x \to 0} \frac{\Delta_x z}{\Delta x}}
    \)

  \item
    по переменной \(y \given
      \display{\frac{\partial z}{\partial y}
      = z'_y
      =\lim_{\Delta y \to 0} \frac{\Delta_y z}{\Delta y}}
    \)
  \end{enumerate}
\end{definition}

\begin{remark}
  При нахождении частной производной можно использовать все свойства производной
  функции одной переменной: другие переменные в этом случае рассматриваются как
  константы. Таким образом частная производная функции нескольких переменных
  определяется также, как и для функции одной переменной, только вместо полного
  приращения функции берется частное приращение по этой переменной. Например,
  частные производные функции \(z = 2 x^2 y^3 + 3x + 5y\) будут равны

  \begin{equation*}
    \frac{\partial z}{\partial x} = 4 y^2 x + 3
    \qquad
    \frac{\partial z}{\partial y} = 6 x^2 y^2 + 5
  \end{equation*}
\end{remark}
