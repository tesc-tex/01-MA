\subsection{%
  Второй замечательный предел. Число \(e\).%
}

\begin{theorem}
  Вторым замечательным пределом называется предел

  \begin{equation*}
    \lim_{x \to \infty} \prh{1 + \frac{1}{x}}^x = e
  \end{equation*}
\end{theorem}

\begin{proof}
  Доказательство этой теоремы делится на два этапа.

  \subsubheader{Шаг I}{Для натуральных чисел}

  Рассмотрим выражение в пределе, раскроем его по биному Ньютона. Обозначив
  \(\display{C_n^k = \frac{n!}{(n - k)! \cdot k!}}\), получим

  \begin{equation*} \label{eq:scd-wond-lim-1} \tag{1}
    \begin{aligned}
      \sum_{k = 0}^{n} \prh{1^{n - k} \cdot \frac{1}{n^k} \cdot C_n^k}
    \\
      = 1 + 1 + \frac{1}{n^2} \cdot \frac{n \cdot (n - 1)}{2!}
        + \frac{1}{n^3} \cdot \frac{n \cdot (n - 1) \cdot (n - 2)}{3!} + \dotsc
    \\
      = 1 + 1 + \frac{1}{2!} \cdot \frac{n - 1}{n}
        + \frac{1}{3!} \cdot \frac{n - 1}{n} \cdot \frac{n - 2}{n} + \dotsc
    \end{aligned}
  \end{equation*}

  Заметим, что эта сумма больше двух. Также заметим, что каждое слагаемое
  положительно, значит последовательность монотонно возрастает. Для того, чтобы
  доказать, что она меньше трёх запишем её в таком виде

  \begin{equation*} \label{eq:scd-wond-lim-2} \tag{2}
    1 + 1 + \frac{1}{2!} \prh{1 - \frac{1}{n}} + \frac{1}{3!} \prh{1 -
      \frac{1}{n}} \prh{1 - \frac{2}{n}} + \dotsc
  \end{equation*}

  Каждая из скобок вида \(\display{\prh{1 - \frac{i}{n}}}\) меньше единицы,
  значит каждое из слагаемых меньше, чем \(\display{\frac{1}{i!}}\). Т.е.
  например \(\display{\frac{1}{3!} \prh{1 - \frac{1}{n}} \prh{1 - \frac{2}{n}} <
  \frac{1}{3!}}\). Теперь заметим, что \(\display{\frac{1}{p!} < \frac{1}{2^{p -
  1}}}\) при \(p > 2\). Из этих двух фактов следует, что сумма
  \eqref{eq:scd-wond-lim-1} меньше чем сумма

  \begin{equation*} \label{eq:scd-wond-lim-3} \tag{3}
    1 + \prh{\frac{1}{2^0} + \frac{1}{2^1} + \frac{1}{2^2} + \dotsc
      + \frac{1}{2^{n - 1}}}
  \end{equation*}

  Сумму большой скобки в \eqref{eq:scd-wond-lim-3} можно вычислить с помощью
  формулы суммы геометрической прогрессии.

  \begin{equation*} \label{eq:scd-wond-lim-4} \tag{4}
    \begin{rcases}
      \sum = \frac{b_1 (1 - q^p)}{1 - q} \\
      b_1 = 1, q = 0.5, p = n
    \end{rcases}
    \implies
    \sum = \frac{1 \cdot (1 - 0.5^n)}{0.5} = 2 - 2 \cdot 0.5^n
  \end{equation*}
  
  В итоге сумма всей правой части в \eqref{eq:scd-wond-lim-3} будет равна \(3 -
  2 \cdot 0.5^n\) (не теряем единичку за большими скобками) Т.к. \(0.5^n \to 0\)
  при \(n \to \infty\), значит эта сумма меньше трёх. Значит искомая сумма
  \eqref{eq:scd-wond-lim-2} также меньше трёх.

  Итого сумма \eqref{eq:scd-wond-lim-2} находится в диапазоне
  \(\interval{2}{3}\). Таким образом последовательность \(\display{\prh{1 +
  \frac{1}{n}}^n}\) ограничена и монотонно возрастает, значит по теореме
  Вейерштрасса она сходится. Число, к которому сходится эта последовательность,
  обозначается \(e\) и примерно равно \(2.718281828\).

  \subsubheader{Шаг II.a}{Для вещественных чисел \(x \to +\infty\)}
  
  Для каждого \(x \in \RR\) найдется такое число \(n \in \NN\), что

  \begin{equation*} \label{eq:scd-wond-lim-5} \tag{5}
    \begin{aligned}
      n \le x < n + 1
    \\
      \frac{1}{n} \ge \frac{1}{x} > \frac{1}{n + 1}
    \\
      \frac{1}{n} + 1 \ge \frac{1}{x} + 1 > \frac{1}{n + 1} + 1
    \end{aligned}
  \end{equation*}

  Запишем исходное неравенство, но в другом порядке: \(n + 1 > x \ge n\).
  Пусть теперь элементы второго неравенства будут показателями степеней для
  элементов из \eqref{eq:scd-wond-lim-5}, тогда мы получим

  \begin{equation*} \label{eq:scd-wond-lim-6} \tag{7}
    \prh{\frac{1}{n} + 1}^{n + 1} > \prh{\frac{1}{x} + 1}^x
      > \prh{\frac{1}{n + 1} + 1}^n
  \end{equation*}

  причем все знаки стали строгими. Рассмотрим предел самого левого элемента
  этого неравенства

  \begin{equation*} \label{eq:scd-wond-lim-8} \tag{8}
    \lim_{n \to \infty} \prh{\frac{1}{n} + 1}^{n + 1}
    = \lim_{n \to \infty} \prh{\frac{1}{n} + 1}^n \cdot \prh{\frac{1}{n} + 1}
    = \lim_{n \to \infty} \prh{\frac{1}{n} + 1}^n
      \cdot \lim_{n \to \infty} \prh{\frac{1}{n} + 1}
  \end{equation*}

  Первый предел равен \(e\) согласно Iому шагу доказательства, второй предел
  равен единице, значит предел левой части равен \(e\) Аналогично предел правой
  части равен \(e\). Значит по теореме о двух жандармах предел выражения в
  центре также равен \(e\), т.е. \(\display{\lim_{x \to \infty}
  \prh{\frac{1}{x} + 1}^x = e}\).

  \subsubheader{Шаг II.a}{Для вещественных чисел \(x \to -\infty\)}

  Сделаем замену \(t = -(x + 1)\), получим, что \(x \to -\infty \implies t \to
  \infty\) и \(x = -(t + 1)\). Подставим это в исследуемый предел.

  \begin{equation*}
    \lim_{t \to \infty} \prh{1 + \frac{1}{-(t + 1)}}^{-(t + 1)}
    = \lim_{t \to \infty} \prh{\frac{t}{t + 1}}^{-(t + 1)}
    = \lim_{t \to \infty} \prh{\frac{t + 1}{t}}^{t + 1}
    = \lim_{t \to \infty} \prh{1 + \frac{1}{t}}^{t + 1}
    = \lim_{t \to \infty} \prh{1 + \frac{1}{t}}^{t} \cdot 
      \lim_{t \to \infty} \prh{1 + \frac{1}{t}}
  \end{equation*}
  
  Пользуясь свойством предела произведения и доказанным на шаге II.a получаем,
  что данный предел равен \(e \cdot 1 = e\).
\end{proof}

\begin{definition}
  Константа \(e = 2.718281828\dotsc\) определяется как значение второго
  замечательного предела.
\end{definition}

\begin{theorem}
  \begin{equation*}
    \log_a (x + 1) \sim \frac{x}{\ln a}
    \qquad
    (x \to 0)
  \end{equation*}
\end{theorem}

\begin{proof}
  Рассмотрим соответствующий предел и перейдем в основанию \(e\) по свойству
  логарифмов.
  
  \begin{equation*}
    \lim_{x \to 0} \frac{\log_a (x + 1)}{\frac{x}{\ln a}}
    = \lim_{x \to 0} \frac{\frac{\ln (x + 1)}{\ln a}}{\frac{x}{\ln a}}
    = \lim_{x \to 0} \frac{\ln (x + 1)}{x}
  \end{equation*}

  По свойству логарифма внесём \(\display{\frac{1}{x}}\) в степень числа
  логарифма, после чего вынесем \(\ln(x)\) за знак предела, т.к. это непрерывная
  функция.

  \begin{equation*}
    \lim_{x \to 0} \frac{\ln (x + 1)}{x}
    = \lim_{x \to 0} \ln \prh{(x + 1)^{1 / x}}
    = \ln \lim_{x \to 0} \prh{(x + 1)^{1 / x}}
  \end{equation*}

  Предел в логарифме является вторым замечательным пределом и равен \(e\),
  значит исходный предел равен \(\ln e = 1\). Таким образом получаем, что
  исследуемые функции эквивалентны при \(x \to 0\) по определению.
\end{proof}

\begin{theorem}
  \begin{equation*}
    a^x - 1 \sim x \ln a
    \qquad
    (x \to 0)
  \end{equation*}
\end{theorem}

\begin{proof}
  Рассмотрим соответствующий предел. Сделаем замену \(y = a^x - 1\), тогда \(x =
  \log_a (y + 1)\), причем \(x \to 0 \implies y \to 0\) (т.к.
  \(\display{\frac{x}{\ln a} \sim \log_a(x + 1)}\)).

  \begin{equation*}
    \lim_{x \to 0} \frac{a^x - 1}{x \ln a}
    = \lim_{y \to 0} \frac{y}{\log_a(y + 1) \ln a}
    = \lim_{y \to 0} \frac{y}{\frac{\ln (y + 1)}{\ln a} \ln a}
    = \lim_{y \to 0} \frac{y}{\ln (y + 1)}
  \end{equation*}
  
  Полученный предел равен единице, т.к. \(y \sim \ln(1 + y)\). Таким образом
  получаем, что исследуемые функции эквивалентны при \(x \to 0\) по определению.  
\end{proof}

\begin{theorem}
  \begin{equation*}
    (1 + x)^n - 1 \sim x n
    \qquad
    (x \to 0)
  \end{equation*}
\end{theorem}

\begin{proof}
  Рассмотрим соответствующий предел, заменим весь числитель на эквивалент по
  правилу \(x \sim \ln(1 + x)\) и упростим

  \begin{equation*}
    \lim_{x \to 0} \frac{(1 + x)^n - 1}{x n}
    = \lim_{x \to 0} \frac{\ln(1 + (1 + x)^n - 1)}{x n}
    = \lim_{x \to 0} \frac{n \ln(1 + x)}{x n}
    = \lim_{x \to 0}{\frac{\ln(1 + x)}{x}}
  \end{equation*}
  
  Полученный предел равен единице, т.к. \(x \sim \ln(1 + x)\). Таким образом
  получаем, что исследуемые функции эквивалентны при \(x \to 0\) по определению.  
\end{proof}
