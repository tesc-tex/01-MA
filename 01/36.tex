\subsection{%
  Формула Тейлора.%
}

\begin{theorem}
  Формулу Тейлора можно обобщить на случай функции нескольких переменных, тогда
  она будет иметь вид

  \begin{equation*}
    z(x, y) = \sum_{k = 0}^n \frac{d^k f(x_0, y_0)}{k!} + r_n(z, x, y)
  \end{equation*}
\end{theorem}

\begin{proof}
  Пусть есть функция \(z = f(x, y)\), которая дифференцируема в окрестности
  точки \(M_0 (x_0, y_0)\). Возьмем точку \((x_0 + \Delta x, y_0 + \Delta y)\)
  из этой окрестности. Выполним линейную параметризацию \(x = x_0 + t \Delta x\)
  и \(y = y_0 + t \Delta y\), причем \(t \in \segment{0}{1}\), получим функцию
  \(g(t) = f(x(t), y(t))\). При линейной параметризации дифференциал сохраняет
  инвариантность формы, а значит \(\dd^n g(t) = \dd^n z(x, y)\).

  В параметризированных уравнениях возьмем производную по \(t\) и умножим на
  соответствующее приращение, чтобы вычислить первые дифференциалы функций одной
  переменной. Получим

  \begin{equation*} \label{eq:T-fml-1} \tag{1}
    \dd x = \Delta x \dd t
    \qquad
    \dd y = \Delta y \dd t
  \end{equation*}

  Запишем формулу Тейлора для функции \(g(t)\) при \(t = t + \Delta t\) и \(t_0
  = t\).

  \begin{equation*} \label{eq:T-fml-2} \tag{2}
    \begin{aligned}
      g(t + \Delta t) = \sum_{k = 0}^n \frac{g^{(k)}(t)}{k!} \cdot (\Delta t)^k
        + \frac{g^{(n + 1)}(\xi)}{(n + 1)!} \cdot (\Delta t)^{n + 1}
      \qquad
      (\xi \in \interval{t}{t + \Delta t})
    \\
      g(t + \Delta t) - g(t) = \sum_{k = 1}^n \frac{d^k g(t)}{k!}
        + \frac{d^{(n + 1)}g(t + \theta \Delta t)}{(n + 1)!}
      \qquad
      (0 < \theta < 1)
    \end{aligned}
  \end{equation*}

  Пусть \(t = 0\), а \(\Delta t = 1\), тогда

  \begin{equation*} \label{eq:T-fml-2} \tag{2}
    \begin{rcases}
      g(t + \Delta t) = g(1) = f(x + \Delta x, y + \Delta y) \\
      g(t) = g(0) = f(x, y)
    \end{rcases}
    \implies
    f(x_0 + \Delta x, y_0 + \Delta y) - f(x_0, y_0)
    = \sum_{k = 1}^n \frac{d^k f(x_0, y_0)}{k!}
      + \frac{d^{(n + 1)} f(x_0 + \theta \Delta t, y_0
      + \theta \Delta t)}{(n + 1)!}
  \end{equation*}

  Причем т.к. \(\Delta t = 1\), то выведенным ранее формулам \eqref{eq:T-fml-1}
  получаем \(\dd x = \Delta x\) и \(\dd y = \Delta y\). Значит дифференциалы
  совпали с зафиксированными вначале приращениями \(\Delta x\) и \(\Delta y\),
  т.е. в этой формуле стоят полные дифференциалы функции \(z(x, y)\). Переписав
  эту формулу в других обозначениях можно получить искомое равенство

  \begin{equation*}
    z(x, y) = \sum_{k = 0}^n \frac{d^k f(x_0, y_0)}{k!} + r_n(z, x, y)
  \end{equation*}
\end{proof}
