\subsection{%
  Первый замечательный предел.%
}

\begin{theorem} \label{thr:fst-wond-lim}
  Первым замечательным пределом называется следующий предел

  \begin{equation*}
    \lim_{x \to 0} \frac{\sin x}{x} = 1
  \end{equation*}
\end{theorem}

\begin{proof}
  Построим единичную окружность и проведем к неё касательную, параллельную оси
  \(Oy\) (\figref{01_09_01}). Выберем на касательной точку и построим
  треугольник \(OAB\). По рисунку видно, что 

  \begin{equation*} \label{eq:fst-wond-lim-1} \tag{1}
    S_{\triangle} OAC < S_{\text{сек}} CA < S_{\triangle} OAB
  \end{equation*}
  
  Выразим эти площади.

  \begin{equation*} \label{eq:fst-wond-lim-2} \tag{2}
    \begin{aligned}
      S_{\triangle} OAC
      = \frac{1}{2} \cdot 1 \cdot 1 \cdot \sin \phi
      = \frac{1}{2} \sin \phi
    \\
      S_{\text{сек}} CA
      = \frac{1}{2} \phi
    \\
    S_{\triangle} OAB
    = \frac{1}{2} \cdot 1 \cdot AB
    = \frac{1}{2} \tg \phi
    & \qquad
    \prh{\tg \phi = \frac{AB}{1}}
    \end{aligned}
  \end{equation*}

  Подставим \eqref{eq:fst-wond-lim-2} в \eqref{eq:fst-wond-lim-1} и преобразуем.

  \begin{equation*} \label{eq:fst-wond-lim-3} \tag{3}
    \begin{aligned}
      \frac{1}{2} \sin \phi < \frac{1}{2} \phi < \frac{1}{2} \tg \phi
    \\
      1 < \frac{\phi}{\sin \phi} < \frac{1}{\cos \phi}
    \\
      1 > \frac{\sin\phi}{\phi} > \cos\phi
    \\
      \lim_{\phi \to 0} 1 > \lim_{\phi \to 0} \frac{\sin\phi}{\phi} >
      \lim_{\phi \to 0} \cos \phi
    \end{aligned}
  \end{equation*}

  Левый и правый пределы равны единице, значит по теореме о двух жандармах
  получаем, что средний предел также равен единице.
\end{proof}

\galleryone{01_09_01}{Первый замечательный предел}

\begin{theorem}
  \begin{equation*}
    x \sim \sin x \qquad (x \to 0)
  \end{equation*}
\end{theorem}

\begin{proof}
  Это следствие из первого замечательного предела (\ref{thr:fst-wond-lim}).
\end{proof}

\begin{theorem}
  \begin{equation*}
    x \sim \arcsin x \qquad (x \to 0)
  \end{equation*}
\end{theorem}

\begin{proof}
  Рассмотрим соответствующий предел.

  \begin{equation*}
    \lim_{x \to 0} \frac{x}{\arcsin x}
    = \mtxb{
      x = \sin t \\
      x \to 0 \implies t \to 0
    }
    = \lim_{t \to 0} \frac{\sin t}{\arcsin (\sin t)}
    = \lim_{t \to 0} \frac{\sin t}{t}
  \end{equation*}

  По первому замечательному пределу это равно \(1\), значит исследуемые функции
  эквивалентны при \(x \to 0\) по определению.
\end{proof}

\begin{theorem}
  \begin{equation*}
    x \sim \tg x \qquad (x \to 0)
  \end{equation*}
\end{theorem}

\begin{proof}
  Рассмотрим соответствующий предел, раскроем тангенс по определению и
  воспользуемся свойствами пределов.

  \begin{equation*}
    \lim_{x \to 0} \frac{\tg x}{x}
    = \lim_{x \to 0} \frac{\sin x}{x \cos x}
    = \lim_{x \to 0} \frac{1}{\cos x} \cdot \lim_{x \to 0} \frac{\sin x}{x}
  \end{equation*}

  Первый предел можно вычислить, он будет равен единице, второй предел также
  будет равен единице согласно первому замечательному пределу. Это значит, что
  и исходный предел также будет равен единице. Таким образом получаем, что
  исследуемые функции эквивалентны при \(x \to 0\) по определению.
\end{proof}

\begin{theorem}
  \begin{equation*}
    x \sim \arctg x \qquad (x \to 0)
  \end{equation*}
\end{theorem}

\begin{proof}
  Рассмотрим соответствующий предел.

  \begin{equation*}
    \lim_{x \to 0} \frac{x}{\arctg x}
    = \mtxb{
      x = \tg t \\
      x \to 0 \implies t \to 0
    }
    = \lim_{t \to 0} \frac{\tg t}{\arctg (\tg t)}
    = \lim_{t \to 0} \frac{\tg t}{t}
  \end{equation*}
  
  Т.к. уже доказано, что \(t \sim \tg t (t \to 0)\), то этот предел равен
  единице, а значит исследуемые функции эквивалентны при \(x \to 0\) по
  определению.
\end{proof}

\begin{theorem}
  \begin{equation*}
    1 - \cos x \sim \frac{x^2}{2} \qquad (x \to 0)
  \end{equation*}
\end{theorem}

\begin{proof}
  Рассмотрим соответствующий предел, в числителе применим формулу повышения
  степени после чего заменим получившийся синус на эквивалент согласно первому
  замечательному пределу.

  \begin{equation*}
    \lim_{x \to 0} \frac{1 - \cos x}{\frac{x^2}{2}}
    = \lim_{x \to 0} \frac{2 \sin^2 \prh{\frac{x}{2}}}{\frac{x^2}{2}}
    = \lim_{x \to 0} \frac{2 \cdot \frac{x^2}{4}}{\frac{x^2}{2}}
    = \lim_{x \to 0} \frac{\frac{x^2}{2}}{\frac{x^2}{2}}
    = 1
  \end{equation*}

  Таким образом получаем, что исследуемые функции эквивалентны при \(x \to 0\)
  по определению.
\end{proof}
