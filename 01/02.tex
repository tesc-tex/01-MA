\subsection{%
  Точка сгущения. Определения предела функции. Односторонние пределы.%
} \label{sec:01-02}

\begin{definition}
  Точка \(a \in \bar{\RR}\) называется точкой сгущения (предельной точкой)
  множества \(E\), если пересечение любой её проколотой \(\delta\)--окрестности
  с этим множеством не пусто.

  \begin{equation*}
    \forall \delta > 0 \given \nearo{a}{\delta} \cap E \neq \varnothing
  \end{equation*}
\end{definition}

\begin{definition} \label{def:lim-C}
  \(L\) называется пределом \textbf{по Коши} функции \(f(x)\) в точке \(a\),
  если

  \begin{equation*}
    L = \lim_{x \to a} f(x) \iff
    \forall \epsilon > 0 \exists \delta > 0 \given
    \forall x \in E \cap \nearo{a}{\delta} \implies
    f(x) \in \near{L}{\epsilon}
  \end{equation*}
\end{definition}

\begin{remark}
  Предел можно определить не только через окрестности, но также и через
  неравенства. \(L\) называется пределом функции \(f(x)\) в точке \(a\), если

  \begin{enumerate}
  \item
    \(
      a \in \RR, L \in \RR \implies
      \lim_{x \to a} f(x) = L \iff
      \forall \epsilon > 0 \exists \delta > 0 \given
      \forall x \colon 0 < \abs{x - a} < \delta \implies
      \abs{f(x) - L} < \epsilon
    \)

  \item
    \(
      a = \pm \infty, L \in \RR \implies
      \lim_{x \to a} f(x) = L \iff
      \forall \epsilon > 0 \exists \delta > 0 \given
      \forall x \colon \abs{x} > \delta \implies
      \abs{f(x) - L} < \epsilon
    \)

  \item
    \(
      a \in \RR, L = \pm \infty \implies
      \lim_{x \to a} f(x) = L \iff
      \forall \epsilon > 0 \exists \delta > 0 \given
      \forall x \colon 0 < \abs{x - a} < \delta \implies
      \abs f(x) > \epsilon
    \)

  \item
    \(
      a = \pm \infty, L = \pm \infty \implies
      \lim_{x \to a} f(x) = L \iff
      \forall \epsilon > 0 \exists \delta > 0 \given
      \forall x \colon \abs{x} > \delta \implies
      \abs f(x) > \epsilon
    \)
  \end{enumerate}
\end{remark}

\begin{remark}
  Можно рассматривать пределы функции в точке только с одной стороны.

  \begin{enumerate}
  \item
    Левосторонний \(
      \lim_{x \to a-0} f(x) = L \iff
      \forall \epsilon > 0 \exists \delta > 0 \given
      \forall x \colon 0 < a - x < \delta \implies
      \abs{f(x) - L} <\epsilon
    \)
  \item
    Правосторонний \(
      \lim_{x \to a+0} f(x) = L \iff
      \forall \epsilon > 0 \exists \delta > 0 \given
      \forall x \colon 0 < x - a < \delta \implies
      \abs{f(x) - L} < \epsilon
    \)
  \end{enumerate}
\end{remark}

\begin{definition} \label{def:lim-G}
  \(L\) называется пределом \textbf{по Гейне} функции \(f(x)\) в точке \(a\),
  если

  \begin{equation*}
    \forall \seq{x_n} \given x_n \to a, x_n \neq a \implies f(x_n) \to L
  \end{equation*}
\end{definition}

Определение \ref{def:lim-G} не было рассмотрено в курсе, однако оно необходимо
для доказательства теорем далее (\ref{sec:01-12}).

\begin{remark}
  С помощью определения предела по Гейне намного проще доказывать то, что предел
  не существует, например рассмотрим предел \(\lim_{x \to \infty} \sin x\).
  Пусть \(\display{a_n = \frac{\pi}{2} + 2 \pi n}\) и \(\display{b_n = \frac{3
  \pi}{2} + 2 \pi n}\).
  
  Обе эти последовательности стремятся к бесконечности при \(n \to \infty\), при
  этом \(\display{\sin \prh{\frac{\pi}{2} + 2 \pi n} = 1}\) и \(\display{\sin
  \prh{\frac{3 \pi}{2} + 2 \pi n} = -1}\). Таким образом предел функции \(\sin
  x\) на бесконечности не существует.
\end{remark}

\begin{theorem}
  Определения \ref{def:lim-C} и \ref{def:lim-G} равносильны.
\end{theorem}

\begin{proof}
  Пусть \(\lim_{x \to a} f(x) = L\).

  \textbf{Коши \(\to\) Гейне}

  Распишем определение предела по Коши.

  \begin{equation*}
    \forall \epsilon_2 > 0 \exists \delta > 0 \given
    \forall x \colon 0 < \abs{x - a} < \delta \implies
    \abs{f(x) - L} < \epsilon_2
  \end{equation*}

  Возьмём произвольную последовательность \(\seq{x_n} \to a, x_n \neq a\). Т.к.
  она стремится к \(a\) (но никогда его не достигает), то по определению предела
  последовательности получаем

  \begin{equation*}
    \forall \epsilon_1 > 0 \exists n_0 \in \NN \given
    \forall n > n_0 \colon 0 < \abs{x_n - a} < \epsilon_1    
  \end{equation*}

  Т.к. это выполняется для любого \(\epsilon_1\), то это выполняется и для
  \(\epsilon_1 = \delta\), а это значит, что

  \begin{equation*}
    \exists n_0 \in \NN \given
    \forall n > n_0 \colon 0 < \abs{x_n - a} < \delta
  \end{equation*}

  Если выполняется неравенство \(0 < \abs{x_n - a} < \delta\), то из определения
  по Коши получаем, что для любого \(\epsilon_2\) выполняется \(\abs{f(x_n) - L}
  < \epsilon_2\). Объединяем все вместе и получаем

  \begin{equation*}
    \forall \epsilon_2 > 0 \exists n_0 \in \NN \given
    \forall n > n_0 \colon \abs{f(x_n) - L} < \epsilon_2    
  \end{equation*}

  Таким образом \(f(x_n) \to L\).

  \textbf{Гейне \(\to\) Коши}

  От противного: пусть предел по Коши не существует, запишем отрицание предела
  по Коши.

  \begin{equation*}
    \exists \epsilon > 0 \forall \delta > 0 \given
    \exists x \colon  0 < \abs{x - a} < \delta \implies
    \abs{f(x) - L} \ge \epsilon
  \end{equation*}

  Т.к. это выполнено для любого \(\delta\), то возьмем \(\display{\delta =
  \frac{1}{n}}\), а \(x\), соответствующий определенному \(\delta\) будем
  называть \(x_n\), получим \(\display{0 < \abs{x_n - a} < \frac{1}{n}}\). Т.к.
  \(n \to \infty\), то по теореме о двух жандармах \(x_n \to a\), но \(x_n \ne
  a\). Таким образом из определения по Гейне получаем, что \(f(x_n) \to L\).
  Однако из отрицания предела по Коши следует, что \(\exists \epsilon > 0 \colon
  \abs{f(x_n) - L} \ge \epsilon\), т.е. \(L\) не является пределом для \(x_n\).
  Противоречие.
\end{proof}

\begin{theorem} \label{thr:lim-unique}
  Если предел функции в точке существует, то он единственный.
\end{theorem}

\begin{proof}
  От противного: пусть \(\lim_{x \to a} f(x) = A\) и \(\lim_{x \to a} f(x) =
  B\), при этом \(A \neq B\). Тогда по определению предела имеем

  \begin{equation*}
    \begin{aligned}
      \forall \epsilon_1 > 0 \exists \delta_1 > 0 \given
      \forall x \in E \cap \nearo{a}{\delta_1} \implies
      f(x) \in \near{A}{\epsilon_1}
    \\     
      \forall \epsilon_2 > 0 \exists \delta_2 > 0 \given
      \forall x \in E \cap \nearo{a}{\delta_2} \implies
      f(x) \in \near{B}{\epsilon_2}
    \end{aligned}
  \end{equation*}
  
  Пусть \(\delta = \min(\delta_1, \delta_2)\), тогда по свойству окрестностей

  \begin{equation*}
    \begin{rcases}
      f(x) \in \near{A}{\epsilon_1} \cap \near{B}{\epsilon_2}
    \\
      \exists \epsilon_1 > 0, \epsilon_2 > 0 \given
        \near{A}{\epsilon_1} \cap \near{B}{\epsilon_2} = \varnothing
    \end{rcases}
    \implies
    f(x) \in \varnothing
  \end{equation*}

  Противоречие.
\end{proof}

\begin{theorem}
  Предел в точке существует \(\iff\) односторонние пределы в этой точке
  существуют и равны.

  \begin{equation*}
    \exists \lim_{x \to a} f(x) = L = A = B
    \iff
    \begin{cases}
      \exists \lim_{x \to a-0} f(x) = A \\
      \exists \lim_{x \to a+0} f(x) = B \\
      A = B
    \end{cases}
  \end{equation*}
\end{theorem}

\begin{proof}
  \ness Распишем односторонние пределы по определению.

  \begin{equation*}
    \begin{aligned}
      \lim_{x \to a-0} f(x) = A \implies
      \forall \epsilon_1 > 0 \exists \delta_1 > 0 \given
      \forall x \colon 0 < a - x < \delta_1 \implies
      \abs{f(x) - A} < \epsilon_1
    \\
      \lim_{x \to a+0} f(x) = B \implies
      \forall \epsilon_2 > 0 \exists \delta_2 > 0 \given
      \forall x \colon 0 < x - a < \delta_2 \implies
      \abs{f(x) - B} < \epsilon_2
    \end{aligned}
  \end{equation*}
  
  Пусть \(\delta = \min(\delta_1, \delta_2)\), тогда по свойству окрестностей
  (учитывая, что \(A = B\)) получаем

  \begin{equation*}
    \begin{rcases}
      \forall \epsilon_1 > 0 \exists \delta > 0 \given
      \forall x \colon 0 < a - x < \delta \implies
      \abs{f(x) - A} < \epsilon_1
    \\
      \forall \epsilon_2 > 0 \exists \delta > 0 \given
      \forall x \colon 0 < x - a < \delta \implies
      \abs{f(x) - A} < \epsilon_2
    \end{rcases}
    \implies
    \forall \epsilon > 0 \exists \delta > 0 \given
    \forall x \colon 0 < \abs{x - a} < \delta \implies
    \abs{f(x) - A} < \epsilon
  \end{equation*}
  
  Это и есть определение предела функции в точке, т.е. \(\lim_{x \to a} f(x) = A
  = B\).
  
  \suff Распишем предел по определению.

  \begin{equation*}
    \lim_{x \to a} f(x) = L \iff
    \forall \epsilon > 0 \exists \delta > 0 \given
    \forall x \colon 0 < \abs{x - a} < \delta \implies
    \abs{f(x) - L} < \epsilon
  \end{equation*}
  
  Раскрывая модуль по определению, получаем \(0 < a - x < \delta\) и \(0 < x - a
  < \delta\). Таким образом оба односторонних предела существуют по определению.
\end{proof}
